\documentclass{amsart}

% Separate paragraphs.
\setlength{\parskip}{\bigskipamount}

% Include AMS math functionality.
\usepackage{amsmath}
\usepackage{amsfonts}
\usepackage{amssymb}
\usepackage{mathrsfs}

% Use hollow squares for list items.
\usepackage{latexsym}
\makeatletter
\AtBeginDocument{
  \def\labelitemi{\(\Box\)}
}

% Diagrams.
\usepackage[all]{xy}

% Enable links in PDF output, making them black and w/ no frame around -- also sections 
% will be displayed in bookmarks pane.
\usepackage[colorlinks,linkcolor=black,citecolor=black]{hyperref}

% Comments -- e.g. \comments{blah, blah}.  Useful for multiline comments as avoids 
% one to write % at the beginning of each line.
\newcommand{\comments}[1]{}



\begin{document}

\title[]{Notes on Counting}
\date{September, 2008}
\author[]{Andrea Falconi}
\keywords{Enumerative combinatorics}

\maketitle

\begin{abstract}
These notes show how to enumerate functions on finite sets and how to relate this
to enumerative combinatorics.
\end{abstract}




\section*{Functions on Finite Sets}
We begin by asking how many ways can can select $r$ elements from a set of $n$ elements,
taking into account the order in which the elements are selected.  That is, we want to know
how many \emph{permutations with repetition} there are.  This is obviously the same as asking
how many sequences of length $r$ are possible from a set of $n$ elements or, equivalently,
what is the cardinality of $Y^{X} = \{f:X \rightarrow Y\}$, when $|X|=r$ and $|Y|=n \ge 1$.

If $X = \emptyset$ ($r = 0$), there is only one possible function $\emptyset \rightarrow Y$;
hence $|Y^{X}| = 1$.

If $X = \{1,\ldots,r,r+1\}$, we notice that, said $Z=\{1,\ldots,r\}$, we can extend every 
$f:Z \rightarrow Y$ into a function $g:X \rightarrow Y$ by letting
\begin{flalign*}
g(x) = \begin{cases}
    f(x) \quad & x \in Z\\
    k    \quad & x = r+1
  \end{cases}
\end{flalign*}

\noindent where $k$ is one of the $n$ elements of $Y$.  Viceversa, every $g: X \rightarrow Y$
is trivially obtained this way.  Since we've got $n$ possible choices for $k$, we also have 
$n$ possible extensions to $f$, therefore $|Y^{X}| = n \cdot |Y^{Z}|$.
For $r=1$, we then obtain $|Y^{X}| = n \cdot 1$; for $r=2$, $|Y^{X}| = n \cdot (n \cdot 1)$;
and so forth.  Thus easily by induction,

\begin{align}
|Y^{X}| = n^{r}
\end{align}

\noindent where $|X| = r \ge 0$, and $|Y| = n \ge 1$.



\section*{Injections}
To continue the above discussion, we now want to determine how many ways can we select 
$r$ \emph{distinct} elements from a set of $n$ elements, again taking into account the order
in which the elements are selected.  That is, we want to determine the number $(n)_{r}$ of
\emph{permutations without repetition}.  Of course, because we require the $r$ elements to be
distinct, this question only makes sense when $r \le n$.  It is trivial to see that this 
question is equivalent to asking how many injections $X \rightarrow Y$ are there when 
$|X|=r$ and $|Y|=n \ge 1$.

We see that to build an injection $f$, we can start out by choosing one element in $Y$ to
associate to $1 \in X$.  Then, if we want $f$ to be injective, we have to send $2$ in one
of the elements of $Y \setminus \{f(1)\}$, so we can only choose from $n-1$ elements.  If
we carry on:
\begin{align*}
&1 \longmapsto f(1) \in Y & \quad \quad \text{$n$ possible choices} \\
&2 \longmapsto f(2) \in Y \setminus \{f(1)\}& \quad \quad \text{$n-1$ possible choices} \\
&3 \longmapsto f(3) \in Y \setminus \{f(1),f(2)\}& \quad \quad \text{$n-2$ possible choices} \\
&\ldots & \ldots \\
&r \longmapsto f(r) \in Y \setminus \{f(1),\ldots,f(r-1)\}& \quad \quad \text{$n-r+1$ possible choices} 
\end{align*}

Hence there are $n \cdot (n-1) \cdot (n-2) \ldots (n-r+1)$ possible ways to build an injection
by using the above procedure.  The same argument can be used to show that any injection $f$ can
actually be built in such way --- $f(1) \in Y$; $f$ injective $\Rightarrow f(2) \in Y \setminus \{f(1)\}$;
and so on.  In conclusion,

\begin{align}
(n)_{r} = |Inj(Y^{X})| = n \cdot (n-1) \cdot (n-2) \ldots (n-r+1) = \frac{n!}{(n-r)!}
\end{align}

\noindent where $|X| = r \ge 0$, $|Y| = n \ge 1$, and $r \le n$.

In the case that $r = n$, then every injection is a bijection too and thus

\begin{align}
|Bi(Y^{X})| = n!
\end{align}




\section*{Factoring through Injections}
What we've discussed so far can be readily used to solve the problem of determining how many
ways can we choose $r$ elements out of $n$.  That is, we're asking what is the number $C_{r}^{n}$ 
of \emph{combinations without repetition}, which is, of course, equivalent to asking how many
subsets of $r$ elements can be formed from $Y$.

Every injective function $f:X \rightarrow Y$ obviously determines a subset of size $r$ of $Y$,
namely $f(X) \subset Y$.  Viceversa, every $S \subset Y$ of size $r$ can be seen as the image
of an injection $f:X \rightarrow Y$ --- if $S = \{s_{1},\ldots,s_{r}\}$, define $f(k) = s_{k}$. 
Hence the image of the function
\begin{align*}
\Phi : Inj(Y^{X}) & \longrightarrow \mathcal{P}(Y) \\
                f & \longmapsto      f(X) 
\end{align*}

\noindent is all the subsets of size $r$ of $Y$; however, a subset $S \subset Y$ of size $r$
may be the image of many injections $X \rightarrow Y$.  How many?  Well, every injection
$X \rightarrow S$ is also a bijection and we know there are $r!$ of those.  This implies the
following relation holds: $|Inj(Y^{X})| =  r! \cdot q$, where $q$ is the number of equivalence
classes in the coimage of $\Phi$
\footnote{That is, the quotient set of the kernel relation of $\Phi$.}
and also happens to be the number of elements in the image of $\Phi$.%
\footnote{Because the coimage of a function is a naturally isomorphic (in the set-theoretic sense
of a bijection) to the image.}
Therefore we have

\begin{align}
C_{r}^{n} = \frac{n \cdot (n-1) \cdot (n-2) \ldots (n-r+1)}{r!} = \frac{n!}{r! \cdot (n-r)!} = 
{n \choose r}
\end{align}

\noindent
A straight consequence of this fact is that $|\mathcal{P}(Y)| = \displaystyle\sum_{r=0}^{n} {n \choose r}$.

\noindent A simple observation is then that the function
\begin{align*}
\chi : \mathcal{P}(Y) & \longrightarrow \mathbf{2}^{Y} \\
                S & \longmapsto      \chi_{S} 
\end{align*}
\noindent defined by
\begin{flalign*}
\chi_{S}(x) = \begin{cases}
    1 \quad & x \in S \\
    0 \quad & x \notin S
  \end{cases}
\end{flalign*}

\noindent is a bijection, therefore if we put all this together, we can conclude

\begin{align}
|\mathcal{P}(Y)| = \displaystyle\sum_{r=0}^{n} {n \choose r} = 2^{n}
\end{align}

\noindent where $|Y| = n \ge 0$.

\end{document}
